\documentclass[a4paper]{article}
\usepackage[czech]{babel}
\usepackage{amsmath}
\usepackage{physics}

\title{Prokládání dat metodou nejmenších čtverců}
\date{2023-04-04}

\begin{document}
\section{Úvod}
\begin{equation}
    (x_i, y_i), \sigma_i \quad i = 1 \dots N
    \quad \text{model}\ y(x, a_1, a_2, \dots a_M)
\end{equation}
\begin{align}
    \chi^2(\vec{a})
    &= \sum_{i = 1}^{N} \left[\frac{y_i - y(x_i, \vec{a})}{\sigma_i}\right]^2
    = \sum_{i = 1}^{N} \Delta^2_i
    \rightarrow \vec{a}_\text{min} \chi^2_\text{min}
    \\
    g_k &= \pdv{\chi^2}{a_k}
    = \sum_{i=1}^{N} -2\Delta_i \frac{1}{\sigma_i}\eval{\pdv{y}{a_k}}_{\vec{a}}
\end{align}

\section{Lineární fitování}
\begin{equation}
    y(x) = \sum_{k=1}^{M} a_k \, \phi_k(x)
\end{equation}

\section{Nelineární fitování}
\subsection{Gaussova-Newtonova metoda}
\begin{equation}
    \vec{a}^{(0)} \rightarrow g_k(\vec{a}^{(0)} + \delta\vec{a})
    = g_k(\vec{a}^{(0)}) + H_{\vec{a^{(0)}}} = 0
\end{equation}
\begin{align}
    \delta\vec{a} &= H^{-1}(-\nabla_{\vec{a}} \chi^2) \\
    \vec{a}^{(1)} &= \vec{a}^{(0)} + \delta\vec{a}
\end{align}
Ukazuje se, že Gaussova-Newtonova metoda nefunguje dobře daleko od minima,
kdy může skončit někde úplně jinde.
Tento problém se snaží odstranit Levenberg-Marquardtův algoritmus.

\subsection{Levenberg-Marquardtův algoritmus}
Je kombinace Gaussovy-Newtonovy metody a metody největšího spádu.
\begin{equation}
    (J^TJ + \lambda \operatorname{diag}J^TJ) \, \delta\vec{a} = -J^T\Delta^T
\end{equation}
Při $\lambda = 0$ tedy krok přechází v~krok Gaussovy-Newtonovy metody,
při $\lambda \gg 1$ zase v~metodu největšího spádu.

Začíná se s~$lambda = 10^{-3}$.
\begin{align}
    \chi^2 (\vec{a} + \delta\vec{a}) &< \chi^2(\vec{a})
    & &\text{$\vec{a}$ přijmeme, $\lambda\ 10\times$ menší} \\
    \chi^2 (\vec{a} + \delta\vec{a}) &> \chi^2(\vec{a})
    & &\text{$\vec{a}$ nepřijmeme, $\lambda\ 10\times$ větší}
\end{align}
Nakonec se spočítá kovarianční matice $C^{-1} = \frac{1}{2} H$.

\section{Nejistoty}
\begin{align}
    \Delta\chi^2 &= \chi^2 - \chi_\text{min}^2 \\
    \Delta\vec{a} &= \vec{a} - \vec{a}_\text{min}^2
    \approx \frac{1}{2}\,\delta\vec{a}^T H \,\delta\vec{a}
    = \delta\vec{a}^T C^{-1} \,\delta\vec{a}
\end{align}
Neznáme-li $\sigma_i$, můžeme je odhadnout:
\begin{equation}
    \sigma = \sqrt{\frac{1}{N - M}(y - y(x_i, \vec{a}_\text{min}))^2}
\end{equation}

\subsection{Kuchařka}
\begin{enumerate}
    \item Vybereme relevantní parametry (počet $k$)
    \item Zvolíme úroveň spolehlivosti P a~z~ní $\Delta{\chi^2}$
    \begin{equation}
        \Delta\chi^2 = F_k^{-1}(P)
    \end{equation}
    \item Promítneme kovarianční matici $C$
    na $k \times k \rightarrow C_\text{proj}$
    \item Vyznačíme oblast s~$\Delta\chi^2 \leq \text{předepsaná}$.
    \begin{equation}
        \Delta\chi^2 \geq \delta a^T C_\text{proj}^{-1} \, \delta\vec{a}
    \end{equation}
\end{enumerate}
\end{document}