\documentclass[a4paper]{article}
\usepackage[czech]{babel}
\usepackage{amsmath}
\usepackage{physics}

\title{Termodynamické simulace metodou Monte Carlo}
\date{2023-05-02}

\newcommand\boltzmann{k_\mathrm{B}}

\begin{document}
\section{Isingův model}
Nejjednodušší model magnetismu.
Nevymyslel ho Ernst Ising, ale jeho školitel.

Mějme síť $L \times L$ bodů, v~každém uzlu je spin
orientovaný nahoru či dolů, $\sigma = \pm 1$.
Celková magnetizace je:
\begin{equation}
    M = \sum_i \sigma_i.
\end{equation}

Máme tedy $L^2$ spinů, počet možných konfigurací je $2^{L^2}$.
Konfiguraci označme $S$, energii této konfigurace $E_S$.

\begin{align}
    H &= -J \sum_{\langle i j \rangle} \sigma_i \sigma_j
    \\
    P(S) &= \frac{1}{Z} e^{-\frac{E_S}{\boltzmann T}}
    \\
    \langle M \rangle &= \sum_S M_S \, P(S)
        = sum_S M_S \frac{1}{Z} e^{-\beta E_S}
\end{align}

Partiční suma:
\begin{equation}
    Z = \sum_S e^{-\beta E_s},
\end{equation}
kde $\beta = \frac{1}{\boltzmann T}$.

Dá se to vyjádřit také pomocí entropie. Volná energie je:
\begin{equation}
    F = U - TS
\end{equation}
Teplota potom vyjadřuje vyvážení mezi vnitřní energií a entropickým členem.

\emph{Importance sampling}: Vybíráme body hustěji v~určitých oblastech.

\subsection{Markovův řetězec}
\begin{equation}
    S_0 \rightarrow S_1 \rightarrow S_2 \rightarrow S_3 \rightarrow \dots
\end{equation}
Procházíme celou mřížku. V~každém kroku se ptáme, zdali převrátit spin.

\subsection{Princip detailního vyvážení}
Pravděpodobnosti výskytu stavu S a S':
\begin{align}
    P(S) &\sim e^{-\beta E_S}
    \\
    P(S') &\sim e^{-\beta E_{S'}}
\end{align}
Přechody mezi stavy $W(S \rightarrow S'), W(S' \rightarrow S)$.

Znění principu:
\begin{equation}
    P(S) W(S \rightarrow S') = P(S') W(S' \rightarrow S).
\end{equation}

\section{Metropolisův algoritmus}
\begin{equation}
    W(S \rightarrow S') = \begin{cases}
        1 & E_{S'} < E_S \\
        \frac{1}{2} & E_{S'} = E_S \\
        e^{-\beta(E_{S'} - E_S)} & E_{S'} > E_S \\
    \end{cases}
\end{equation}

\end{document}
