\documentclass[a4paper]{article}
\usepackage[czech]{babel}
\usepackage{amsmath}
\usepackage{physics}

\title{Simulace metodou Monte Carlo}
\date{2023-04-25}

\begin{document}
Jednoduché simulace pomocí metody Monte Carlo:
radioaktivní rozpad, náhodné procházky.

\section{Radioaktivní rozpad}
Simulujeme známý výsledek:
\begin{equation}
    N(t) = N(0) \, e^{-\frac{t}{\tau}},
\end{equation}
kde $\tau$ je charakteristický čas rozpadu.

Začneme s~$N$ jádry a~v~každém kroku pro každé jádro určíme,
zdali se rozpadne, či ne.
Pravděpodobnost rozpadu za $\dd{t}$ je:
\begin{equation}
    P(\text{rozpad za dt}) = \frac{\dd{t}}{\tau}.
\end{equation}

\section{Náhodná procházka}
\subsection{Náhodné kráčení v~1D s~pevným krokem}
Začneme v~určitém bodě a~v~každém kroku se rozhodneme,
zdali se vydáme doleva či doprava.
Rozdělovací funkce navštívených bodů má šířku úměrnou $\sqrt{N}$.

\subsection{Náhodné kráčení ve~2D po mřížce}
Zase nějaká gaussovka.

\subsection{Náhodná procházka ve~2D s~pevným krokem a~náhod\-ným směrem}
\newcommand\sumN{\sum_{i=1}^N}
\newcommand\mean[1]{\langle #1 \rangle}
Náhodný směr je dán $\varphi = 2 \pi r$, počet kroků je $N$.
\begin{equation}
    \{\varphi_i\}_{i=1}^N \rightarrow
\end{equation}
Polohu koncového bodu po $N$ krocích označme $\vec r$.
\begin{equation}
    \vec{r}_N = \left(
        \sum_{i=1}^N \cos\varphi_i, \sum_{i=1}^N \sin\varphi_i
        \right).
    \end{equation}
Střední vzdálenost koncového bodu od počátku je:
\begin{align}
    \langle r_N^2 \rangle &= \langle
        \left(\sum_{i=1}^N \cos\varphi_i\right)^2
        + \left(\sum_{i=1}^N \sin\varphi_i\right)^2
    \rangle
    \\
    &= \langle \sumN \cos^2 \varphi_i
        + \sum_i\sum_{j \ne i} \cos\varphi_i\cos\varphi_j
        + \sumN \sin^2 \varphi_i
        + \sum_i\sum_{j \ne i} \sin\varphi_i\sin\varphi_j
    \\
    &= \sumN \mean{\cos^2 \varphi_i}
        + \sum_i\sum_{j \ne i} \mean{\cos\varphi_i\cos\varphi_j}
        + \sumN \mean{\sin^2 \varphi_i}
        + \sum_i\sum_{j \ne i} \mean{\sin\varphi_i\sin\varphi_j}
    \\
    &= N \mean{\cos^2\varphi + \cos^2\varphi}
    \\
    &= N
\end{align}

(Dále byla počítána poloha gyračního poloměru.)

\subsubsection{Rozdělení poloh koncových bodů}
\begin{align}
    x_N &= \sumN \cos\varphi_i
    \\
    \mean{\cos\varphi_i} &= 0
    \\
    \sigma^2 &= \mean{cos^2\varphi}
        = \frac{1}{2\pi} \int_0^{2\pi} \cos^2\varphi \dd{\varphi} = \frac12
    \\
    \sigma &= \frac{1}{\sqrt{2}}
\end{align}

%Podle centrální limitní věty vyjde gaussovo rozdělení s~$\mu = 0$
%~$\sigma = \frac{}{}.
\end{document}
